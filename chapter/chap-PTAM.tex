
\def \R2{\mathbb{R}^2}
\def \R3{\mathbb{R}^3}
\def \Rn{\mathbb{R}^n}

\def \itW{\mathit{W}}

\def \itK{\mathit{K}}
\def \bfp{\mathbf{p}}

\chapter{基于单目相机的3维实时重建的PTAM算法介绍}

\section{数学知识介绍}

\section{PTAM算法的介绍}
关键词,双管线系统,为后面的算法奠定了基础

\section{PTAM算法的数学推导}
我们首先简述PTAM实现单目相机SLAM的原理。单目相机模型不同于双目相机,实时追踪时相机视界中的点不能和其它相机视界中的点进行匹配,只能和自己的关键帧匹配,从而加大了3D重建中定位的难度。PTAM算法提出利用单目相机实时追踪特征点的可行性,实现三维重建。

\subsection{PTAM算法工作的基本流程}
PTAM算法主要思想是将Tracking和Mapping两个过程放在不同的管线(进程)中进行: Tracking 进程专门实现相机位置的估计,Mapping 进程则用于进行关键帧之间的误差消除。

如果记$\itW$为真实世界的坐标系,PTAM算法将维护一个关键帧集合: $Img=\{I_1,I_2,\ldots,I_m\}$,这\(m\)个关键帧分别对应\(m\)个相机坐标系 $\itK_i$。我们用 $E_{\itK_i\itW}$ 表示从世界坐标系到相机坐标系的仿射变换(Affine Transformation)。

\subsection{PTAM管线之一: Tracking}

%
% Todo: 插入一张流程图
%
追踪进程需要解决如下的问题:

当读入了新的关键帧之后,原来算法在重建过程中提取的3维空间中的特征点现在在照片中的坐标是什么?现在的相机姿态应该怎么估计? 

我们假定程序可以从映射进程得到一个关键帧集合 $Img$ 以及3维重建的特征点的坐标集合(相对于世界坐标系)$P=P_\itW=\{\bfp_{1\itW},\ldots,\bfp_{s\itW}\}$,为了统一形式,将第$j$个点坐标记为 $\bfp_{j\itW}= (p_{jx},p_{jy},p_{jz},1)$。

根据已知结论,仿射坐标系变换对应公式为

\begin{equation}
\bfp_{jK_{t+1}}= E_{K_{t+1}\itW} \bfp_{j\itW}
\end{equation}

为了把三维空间中的视界投影到二维空间,算法遵循\citep{deng:01a}中的FOV相机模型,构建一个$\mathbb{R}^3$到 $\mathbb{R}^2$ 的映射为:

\begin{equation}
\label{eq:FOV}
f(x,y,z,1)=(u_0,v_0) + (x/z,y/z) 
\begin{bmatrix}
       f_u  & 0 \\
       0 & f_v  \\
\end{bmatrix} \frac{r'}{r}
\end{equation}
\begin{equation}
r= \sqrt{\frac{x^2+y^2}{z^2}}
\end{equation}

\begin{equation}
r'= \frac{1}{\omega} arctan(2rtan\frac{\omega}{2})
\end{equation}

其中我们假定焦距$f_u,f_v$,主点位置$(u_0,v_0)$和畸变系数$\omega$已知。这时对于实时相机姿态的更新,相当于对于 \autoref{eq:FOV} 求微分,在假定线性运动的情况下更新相机姿态的变换仿射矩阵。若我们设从上一关键帧到下一个关键帧的仿射变换矩阵为$T$,则有以下的关系成立:

\begin{equation}
E_{K_{t+1}\itW}}  = T E_{K_{t}\itW}} = exp(\mu) E_{K_{t}\itW}}
\end{equation}

其中$\mu$为6维向量,代表矩阵 $T$的6个自由度。所以问题转化为: 根据图像中特征点$(u,v)$的变化和在每个特征点三维空间中的预估位置,求解$\mu,T$的取值。


\subsubsection{关键特征点的投影和配对}


我们假定点$p$为我们需要定位的特征点,则我们首先需要在当前的关键帧图像中找到该点的新投影坐标$(\hat{u},\hat{v})$。为此,我们构造图像的尺度空间,利用 %\citep{}
的FAST计算角点的方法提取可能的特征点。在假定相机移动很慢的情况下,特征点匹配算法从上一帧的位置开始,在一定的半径阈值内,在对应的视界空间上进行搜索。







\subsubsection{相机模型参数的求解和位置的更新}


为了保证算法的稳定性,追踪进程会进行两次:第一次会从三维模型中抽取50个特征点投影匹配,第二次则抽取1000个特征点进行匹配。


\subsection{PTAM管线之二: Mapping}

\subsubsection{关键帧的选择和插入}

\subsubsection{利用Bundle Adjustment极小化误差}


\section{PTAM算法的实际应用}
\subsection{在一般电脑上运行}
\subsection{在智能手机上运行}

\section{PTAM算法的评价}
