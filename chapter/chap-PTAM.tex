
\def \R2{\mathbb{R}^2}
\def \R3{\mathbb{R}^3}
\def \Rn{\mathbb{R}^n}

\def \itW{\mathit{W}}

\def \itK{\mathit{K}}
\def \bfp{\mathbf{p}}

\chapter{基于单目相机的3维实时重建的PTAM算法介绍}


\section{PTAM算法的介绍}
关键词,双管线系统,为后面的算法奠定了基础

我们首先简述PTAM实现单目相机SLAM的原理。单目相机模型不同于双目相机模型,实时追踪时相机视界中的点不能和其它相机视界中的点进行匹配,只能和自己的关键帧匹配,从而加大了3D重建中定位的难度。PTAM算法提出利用单目相机实时追踪特征点的可行性,实现三维重建。

PTAM算法主要思想是将Tracking和Mapping两个过程放在不同的管线(进程)中进行: Tracking 进程专门实现相机位置的估计,Mapping 进程则用于进行关键帧之间的误差消除。

\section{PTAM算法工作的基本流程}
如果记$\itW$为真实世界的坐标系,PTAM算法将维护一个关键帧集合: $Img=\{I_1,I_2,\ldots,I_m\}$,这\(m\)个关键帧分别对应\(m\)个相机坐标系 $\itK_i$,以及一个三维重建的结果。我们用 $E_{\itK_i\itW}$ 表示从世界坐标系到相机坐标系的仿射变换(Affine Transformation)。

\subsection{PTAM管线之一: Tracking}

%
% Todo: 插入一张流程图
%
追踪进程需要解决如下的问题:

当读入了新的关键帧之后,原来算法在重建过程中提取的3维空间中的特征点现在在照片中的坐标是什么?现在的相机姿态应该怎么估计? 

我们假定程序可以从映射进程得到一个关键帧集合 $Img$ 以及3维重建的特征点的坐标集合(相对于世界坐标系)$P=P_\itW=\{\bfp_{1\itW},\ldots,\bfp_{s\itW}\}$,为了统一形式,将第$j$个点坐标记为 $\bfp_{j\itW}= (p_{jx},p_{jy},p_{jz},1)$。

根据已知结论,仿射坐标系变换对应公式为

\begin{equation}
\bfp_{jK_{t+1}}= E_{K_{t+1}\itW} \bfp_{j\itW}
\end{equation}

为了把三维空间中的视界投影到二维空间,算法遵循\citep{deng:01a}中的FOV相机模型,构建一个$\mathbb{R}^3$到 $\mathbb{R}^2$ 的映射为:

\begin{equation}
\label{eq:FOV}
f(x,y,z,1)=(u_0,v_0) + (x/z,y/z) 
\begin{bmatrix}
       f_u  & 0 \\
       0 & f_v  \\
\end{bmatrix} \frac{r'}{r}
\end{equation}
\begin{equation}
r= \sqrt{\frac{x^2+y^2}{z^2}}
\end{equation}

\begin{equation}
r'= \frac{1}{\omega} arctan(2rtan\frac{\omega}{2})
\end{equation}

其中我们假定焦距$f_u,f_v$,主点位置$(u_0,v_0)$和畸变系数$\omega$已知。这时对于实时相机姿态的更新,相当于对于 \autoref{eq:FOV} 求微分,在假定线性运动的情况下更新相机姿态的变换仿射矩阵。若我们设从上一关键帧到下一个关键帧的仿射变换矩阵为$T$,则有以下的关系成立:

\begin{equation}
E_{K_{t+1}\itW}  = T E_{K_{t}\itW} = exp(\mu) E_{K_{t}\itW}
\end{equation}

其中$\mu$为6维向量,代表矩阵 $T$的6个自由度。所以问题转化为: 根据图像中特征点$(u,v)$的变化和在每个特征点三维空间中的预估位置,求解$\mu,T$的取值。


我们假定点$p$为我们需要定位的特征点,则我们首先需要在当前的关键帧图像中找到该点的新投影坐标$(\hat{u},\hat{v})$。为此,我们构造图像的尺度空间,利用 %\citep{}
的FAST计算角点的方法提取可能的特征点。在假定相机移动很慢的情况下,特征点匹配算法从上一帧的位置开始,在一定的半径阈值内,在对应的视界空间上,依据设计好的评价函数(scoring function)进行搜索。
算法会在搜索过程中得到一个坐标$(\hat{u},\hat{v})$,以及 $\sigma=2^l$ 作为视界空间金字塔中搜索对应的层数。在得到所有特征点的信息之后,算法求解以下的优化问题
计算相机姿态的更新(设$P_{t+1}$为当前特征点集合):

\begin{equation}
\underset{\mu}{argmin} \sum_{j\in P_{t+1}} \psi( \frac{\lVert\mathbf{e}_j\rVert}{\sigma_j} , \sigma_T)
\end{equation}
\begin{equation}
\mathbf{e}_j= (\hat{u_j},\hat{v_j})- f(exp(\mu)E_{K_{t}\itW} \mathbf{p_j})
\end{equation}

这里 $\psi$ 为 Tukey loss:

\begin{equation}
\psi(x,c)= 
\begin{cases}
x(1-\frac{x^2}{c^2}) & \text{for }|x|<c \\
0  &  \text{for }|x|>c
\end{cases}
\end{equation}

为了保证算法的稳定性,追踪进程会进行两次:第一次会从三维模型中抽取50个特征点投影匹配,第二次则抽取1000个特征点进行匹配。同时,如果在特征点投影搜索配对的过程中,如果特征点搜索失败的比率大于某一个阈值的话,算法将会判定这一帧失效,并且自动舍弃(这可能是由于抖动,位移量过大特征点不在视界中等多种原因造成)。


\subsection{PTAM管线之二: Mapping}

如图 %\autoref{??}
所示在追踪进程持续运行的同时,映射进程会根据追踪进程估计的相机姿态,完成关键帧的选取及三维重建的主要任务。为此,算法首先在初始化阶段,需要用户指定视频流中的两帧(第一帧一般为起始帧),单独进行第一次特征点配对。这样做的目的有两个,一方面是为了利用五点法求解相机模型中的固有参数,另一方面是为了先定位特征点到三维空间中作为初始信息。

\subsubsection{关键帧的选择和插入}
在程序进程中,当视频流被读入之后,对于每一帧图像,算法需要判断是否需要把该帧图像加入关键帧集合。判断主要依据是距离上一关键帧的时间和预估的相机距离,以及图像的质量。在选定关键帧之后,根据追踪进程返回的信息,结合上一关键帧的对应特征点信息,定位深度,从而把特征点加入到三维重建模型中。

\subsubsection{利用BA(Bundle Adjustment)校正}
为了消除帧与帧之间的积累误差,映射进程还需要对整体进行一个调整,即所谓的BA。算法对于所有关键帧求解一次优化问题,也会对局部的几个特征点求解局部的优化问题,它们有以下形式:

\begin{equation}
\underset{\{\mu\},\{\mathbf{p}\}}{argmin} = \sum_{i=1}^{N} \sum_{j\in P_i} \psi(\frac{\lVert\mathbf{e}_j\rVert}{\sigma_j},\sigma_T)
\end{equation}

在完成校正之后,新的特征点的三维信息和关键帧信息将作为追踪进程的输入,循环使用。



\section{PTAM算法的实际应用}
\subsection{在一般电脑上运行}

\subsection{在智能手机上运行}

\section{PTAM算法的评价}
