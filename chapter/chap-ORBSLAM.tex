\chapter{ORB-SLAM在家居设计和增强现实中的应用构想}

\section{ORB-SLAM方法简述}

ORB-SLAM \cite{Mur-Artal2015}
是近几年比较火热的单目相机的SLAM方法,它在PTAM的基础上采用了效率更高的特征提取和优化算法,改进了PTAM的双进程为三进程,加入闭环检测(loop closing),并优化了代码结构,使得该算法较PTAM更稳定,更实用,也更有移植到移动端的潜力。

该算法的流程如\autoref*{fig:ORBprocess}所示,算法主要维护三个进程:追踪进程(Tracking)与PTAM类似,主要用于相机姿态的估计,但不同的是采用了比FAST更优的ORB \cite{Strasdat2011}
特征点提取方法,同时改进了关键帧的判定。局部映射进程(Local Mapping)则与PTAM中的映射进程类似,在有了新的关键帧之后,算法需要维护关键帧集合,构建重建信息,同时进行BA优化。ORB-SLAM算法维护的第三个进程,即闭环检测进程(loop closing),利用了DBoW2\cite{Galvez-Lopez2012}技术,通过构建词典检测图片序列中的语义场景,从而判定相机轨迹中是否存在闭路的情形,
同时对检测到的闭路进行优化,提高追踪精度并减少重复计算。

为了降低全局的时间复杂度,ORB-SLAM维护一个基于关键帧到关键帧的图结构(Covisibility Graph)
\cite{Galvez-Lopez2012}
,用图模型中边的权值(共享特征点数量)衡量关键帧的相似度,同时取稠密图中的MST(Essential Graph)作为子图实际参与优化,在保证鲁棒性的前提下尽可能提高效率。


\begin{figure}[!htbp]
\centering
\includegraphics[width=12cm]{ORBSLAM.pdf}
\caption{ORB-SLAM算法的主要流程示意}
\label{fig:ORBprocess}
\end{figure}

\section{ORB-SLAM方法评价}
ORB-SLAM基本延续了 PTAM 的算法框架,但对框架中的大部分组件都做了改进, 归纳起来主要有 4 点: \cite{刘浩敏2016基于单目视觉的同时定位与地图构建方法综述}

(1)ORB-SLAM选用了ORB特征,基于ORB描述量的特征匹配和重定位,都比PTAM具有更好的视角不变性。此外,新增三维点的特征匹配效率更高,因此能更及时地扩展场景。扩展场景及时与否决定了后续帧是否能稳定跟踪。

(2)ORBSLAM加入了循环回路的检测和闭合机制, 以消除误差累积. 系统采用与重定位相同的方法来检测回路(匹配回路两侧关键帧上的公共点),通过方位图(Pose Graph)优化来闭合回路。

(3)PTAM需要用户指定2帧来初始化系统,2帧间既要有足够的公共点,又要有足够的平移量。平移运动为这些公共点提供视差(Parallax),只有足够的视差才能三角化出精确的三维位置。ORB-SLAM 通过检测视差来自动选择初始化的2帧。

(4)PTAM扩展场景时也要求新加入的关键帧提供足够的视差,导致场景往往难以扩展。ORB-SLAM采用一种更鲁棒的关键帧和三维点的选择机制——先用宽松的判断条件尽可能及时地加入新的关键帧和三维点,以保证后续帧的鲁棒跟踪,再用严格的判断条件删除冗余的关键帧和不稳定的三维点, 以保证 BA 的效率和精度。

总之,ORB在理论开创性方面远不及PTAM,然而ORB也绝不仅仅是PTAM添加了loop closure模块并替换图像特征为ORB而已。它吸收了近几年monoslam领域的很多理论成果,比如逆深度的使用,g2o工具箱的优化等。
\section{ORB-SLAM方法在家具设计与增强现实中的应用构想}
知名家居公司宜家(Ikra)于2014年推出了增强现实app,可以通过扫描目录页,而后把产品目录放在想要摆放家具的位置上,而后就可以在自己的手机或者平板电脑的屏幕上看到用户想要看到的家具放在自己家里的模样,其效果图如\autoref*{fig:yijia_app}所示。这种把家具拉出来与实际的居家环境结合的增强现实技术,带给了使用者更好的购物体验,而这也给宜家带来了巨大的收益。

\begin{figure}[!htbp]
\centering
\includegraphics[width=12cm]{yijia_app.jpg}
\caption{Ikra app示意图}
\label{fig:yijia_app}
\end{figure}

然而,这个app有一个很大的缺点,那就是使用这个app需要购买宜家公司的产品目录,而其app采用的同时定位与地图构建方法则通过直接使用产品目录的位置大大简化了流程,即不再需要跟踪寻找关键帧,只需要跟踪产品目录的位置即可。这一行为于宜家公司来说绑定了产品目录的销量,另一方面却给用户带来了很多不便。

因此,我们考虑使用ORB-SLAM技术对其进行改良。

