\chapter{动态环境中的SLAM算法介绍}
实际应用中我们可能会遇到对于拍摄的物体有运动的情况,此时我们首先要区分出静态物体和动态物体,进一步再刻画两者的位置关系,两者的位置需要交错迭代来确认。另外为了提高位置的准确性,我们还需要提供标定点来给相机参考\cite{wang2002simultaneous},或由相机位置计算出标定点\cite{wang2003online},这里我们采用角落作为标定点。过程中使用两个集合$S$和$D$,分别表示格点被静态部分和动态部分占用的概率,$S \cup D$ 则可表示出可推测部分的概率。当$S$或$D$中的某一个对应概率超过$0.9$时,我们认为该位置是可以确定的。
\section{物体类型的映射的迭代}
记 $o_t$ 为 $t$ 时刻物体位置的深度信息,则我们希望求出 $P(S_t|o_1,…,o_t)$, $P(D_t|o_1,…,o_t)$。 为了得到迭代过程,我们期望$P(S_t)$ 和 $P(D_t)$ 依赖于 $S_{t-1}, D_{t-1}$, 而由于$D_{t-1}$ 中信息会发生变化,我们在迭代过程中不利用$D_{t-1}$ 中的信息。即我们希望求出 $P(S_t|o_1,…,o_t,S_{t-1})$, $P(D_t|o_1,…,o_t,S_{t-1})$.
\section{映射集S的迭代}
利用Bayes 公式, 可以得到
\begin{equation*}
P(S_t|o_1,\cdots,o_t,S_{t-1}) = \dfrac{P(o_t|o_1,\cdots,o_{t-1},S_{t-1},S_t)P(S_t|o_1,\cdots,o_{t-1},S_{t-1})}{P(o_t|o_1,\cdots,o_{t-1},S_{t-1})}
\end{equation*}
在 $P(o_t|o_1,\cdots,o_{t_1},S_{t_1},S_t)$ 中,我们考虑 $S_{t-1}, S_{t}$ 已经包含了 $\{o_1,\cdots,o_t\}$ 中的所有信息,即 $P(o_t|o_1,\cdots,o_{t-1},S_{t-1},S_t) = P(o_t|S_{t-1},S_t)$,再次利用Bayes 公式,可得
\begin{equation*}
P(S_t|o_1,\dots,o_t,S_{t-1}) = \dfrac{P(S_t|o_t,S_{t-1})P(o_t|S_{t-1})P(S_t|o_1,\cdots,o_{t-1},S_{t-1})}{P(S_t|S_{t-1})P(o_t|o_1,\cdots,o_{t-1},S_{t-1})}
\end{equation*}
类似可以求出$S$未占据格点个条件概率$P(\widetilde{S})$,
\begin{equation*}
P(\widetilde{S_t}|o_1,\dots,o_t,S_{t-1}) = \dfrac{P(\widetilde{S_t}|o_t,S_{t-1})P(o_t|S_{t-1})P(\widetilde{S_t}|o_1,\cdots,o_{t-1},S_{t-1})}{P(\widetilde{S_t}|S_{t-1})P(o_t|o_1,\cdots,o_{t-1},S_{t-1})}
\end{equation*}
上两式相除,可得
\begin{equation*}
\dfrac{P(S_t|o_1,\dots,o_t,S_{t-1})}{P(\widetilde{S_t}|o_1,\dots,o_t,S_{t-1})} = \dfrac{P(S_t|o_t,S_{t-1})}{P(\widetilde{S_t}|o_t,S_{t-1})}\dfrac{P(\widetilde{S_t}|S_{t-1})}{P(S_t|S_{t-1})}\dfrac{P(S_t|o_1,\cdots,o_{t-1},S_{t-1})}{P(\widetilde{S_t}|o_1,\cdots,o_{t-1},S_{t-1})}
\end{equation*}
由于$P(S) = 1 - P(S')$, 上式可写为
\begin{equation*}
\dfrac{P(S_t|o_1,\dots,o_t,S_{t-1})}{1-P({S_t}|o_1,\dots,o_t,S_{t-1})} = \dfrac{P(S_t|o_t,S_{t-1})}{1-P({S_t}|o_t,S_{t-1})}\dfrac{1-P({S_t}|S_{t-1})}{P(S_t|S_{t-1})}\dfrac{P(S_t|o_1,\cdots,o_{t-1},S_{t-1})}{1-P({S_t}|o_1,\cdots,o_{t-1},S_{t-1})}
\end{equation*}
由于$P(S_t|S_{t-1})$ 不依赖于$t$, 可记为 $P(S)$, 另外,由于$S^t$ 在给定 $o_1,\cdots,o_{t-1},S_{t-1}$时,不发生改变,因此上式可写为
\begin{equation*}
\dfrac{P(S_t|o_1,\dots,o_t,S_{t-1})}{1-P({S_t}|o_1,\dots,o_t,S_{t-1})} = \dfrac{P(S_t|o_t,S_{t-1})}{1-P({S_t}|o_t,S_{t-1})}\dfrac{1-P(S)}{P(S)}\dfrac{P(S_{t-1})}{1-P(S_{t-1})}
\end{equation*}
为便于计算,上式中取$P(S) = 0.5$,可得
\begin{equation*}
\dfrac{P(S_t|o_1,\dots,o_t,S_{t-1})}{1-P({S_t}|o_1,\dots,o_t,S_{t-1})} = \dfrac{P(S_t|o_t,S_{t-1})}{1-P({S_t}|o_t,S_{t-1})}\dfrac{P(S_{t-1})}{1-P(S_{t-1})}
\end{equation*}
上式即可完成迭代过程。
\section{映射集D的迭代}
类似于$S$的计算过程,我们可以得到
\begin{equation*}
\dfrac{P(D_t|o_1,\dots,o_t,S_{t-1})}{1-P({D_t}|o_1,\dots,o_t,S_{t-1})} = \dfrac{P(D_t|o_t,S_{t-1})}{1-P({D_t}|o_t,S_{t-1})}\dfrac{P(D_{t-1})}{1-P(D_{t-1})}
\end{equation*}
由于$D$中的物体会随时间变化,迭代过程不需要依赖于$D_{t-1}$, 计算时也不需要考虑 $D$ 中的过去值,而只需要考虑当前的$t$ 来计算当前移动物体的位置情况。仅当$P(S_{t-1})$未确定且$P{o_t}$较大时,$P(D_t)$ 会有较大的取值。\cite{wolf2005mobile}
\section{定位}
由于动态物体位置的确认依赖于原来区域的选取,因此静态位置误差会导致对动态物体估计的误差。位置标定的方法有很多种算法,如我们下面使用的地标法。若对环境有预先的信息,则这里可以在给定的信息中得到标定点,否则,要同时利用相机位置和图片信息\cite{dissanayake2001solution}, 可以选取角作为标定点\cite{tomasi1991detection},为了避免标定点的相似影响结果,还需要相机的位置来辅助确定位置,并删去最近的标定点。这里除了上述的两种格点映射集,我们还需要第三类映射集来储存标定点的位置。同样,标定点位置可能会被动态物体影响,特别是将动态物体设置为标定点时,因此必须区分出静态标定点。我们在选取标定点时,必要的要从S中大概率的点来选取。
