\chapter{总结和展望}
本文首先简述了基于视觉的同时定位与地图构建的基本原理并比较了不同单目同时定位与地图构建技术的优劣,同时详细介绍了基于关键帧的单目相机实施重建算法,同时对ORB-SLAM算法进行了改进,并提出了几点在家具设计和增强现实中的应用构想。

近年来,移动终端、头戴设备的快速发展为增强现实技术提供了很好的硬件展示平台,而SLAM作为增强现实的关键基础技术,近年来也得到了快速发展。而单目相机也逐渐被多目相机乃至深度相机所取代。随着各种硬件传感器的发展和普及,目前SLAM技术正朝着多传感器融合的方向发展,试图通过利用各种传感器的优势互补性来达到尽可能高的精度和鲁棒性。

对于移动增强现实应用来说,由于通常采用的传感器、CPU等硬件设备的性能受限于价格、功耗等因素,这就要求SLAM算法具有很高的鲁棒性和计算效率,如何进一步提高SLAM算法的鲁棒性和计算效率,通过软件算法节约硬件成本,将会是一个很有价值的研究方向。

同时,在实际应用中,由于会受到环境,设备等影响,(例如对于动态环境来说SLAM算法表现不佳),如何提高SLAM的鲁棒性,使得其能应对各种各样地复杂情况,这是接下来一个很好的研究方向。