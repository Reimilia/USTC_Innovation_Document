\documentclass[12pt]{article}
\usepackage{amssymb}
\usepackage{mathrsfs}
%\addtolength{\topmargin}{-.3in} \addtolength{\textheight}{.5in}
%\addtolength{\oddsidemargin}{-.3in}
%\addtolength{\evensidemargin}{-.3in} \addtolength{\textwidth}{0.6in}
\usepackage{latexsym,amsmath,amssymb,amsfonts,epsfig,graphicx,cite,psfrag}
\usepackage{eepic,color,amscd}
\usepackage[colorlinks,linkcolor=blue]{hyperref}
\usepackage{changepage}
\usepackage{enumitem}
%\usepackage{tabu} 
\usepackage{subfigure}
\usepackage{ctex}
\DeclareGraphicsExtensions{.eps,.ps,.png,.jpg,.bmp}

\pagestyle{plain}

\begin{document}
\begin{center}
PTAM(\textbf{Parallel Tracking and Mapping})算法综述
\end{center}
\section{PTAM算法的介绍}
关键词,双管线系统,为后面的算法奠定了基础

\section{PTAM算法的数学推导}
我们首先简述PTAM实现单目相机SLAM的原理。单目相机模型不同于双目相机,实时追踪时相机视界中的点不能和其它相机视界中的点进行匹配,只能和自己的关键帧匹配,从而加大了3D重建中定位的难度。PTAM算法提出利用单目相机实时追踪特征点的可行性,实现三维重建。

\subsection{PTAM算法工作的基本流程}
PTAM算法主要思想是将Tracking和Mapping两个过程放在不同的管线(进程)中进行: Tracking 进程专门实现相机位置的估计,Mapping 进程则用于进行关键帧之间的误差消除。

\subsection{PTAM管线之一: Tracking}


\subsubsection{关键特征点的匹配和搜索}

\subsubsection{相机模型参数的求解和位置的更新}


\subsection{PTAM管线之二: Mapping}

\subsubsection{关键帧的选择和插入}

\subsubsection{利用Bundle Adjustment极小化误差}


\section{PTAM算法的实际应用}
\subsection{在一般电脑上运行}
\subsection{在智能手机上运行}

\section{PTAM算法的评价}

\end{document}