\documentclass[14pt]{article}
\usepackage{amssymb}
\usepackage{mathrsfs}
\usepackage{stmaryrd}
\usepackage{amsmath}
\DeclareMathOperator*{\argmax}{argmax}
%\addtolength{\topmargin}{-.3in} \addtolength{\textheight}{.5in}
%\addtolength{\oddsidemargin}{-.3in}
%\addtolength{\evensidemargin}{-.3in} \addtolength{\textwidth}{0.6in}
\usepackage{latexsym,amsmath,amssymb,amsfonts,epsfig,graphicx,cite,psfrag}
\usepackage{eepic,color,amscd}
\usepackage[colorlinks,linkcolor=blue]{hyperref}
\usepackage{changepage}
\usepackage{enumitem}
%\usepackage{tabu} 
\usepackage{subfigure}
\usepackage{ctex}
\DeclareGraphicsExtensions{.eps,.ps,.png,.jpg,.bmp}

\pagestyle{plain}

\begin{document}
\begin{center}
\textbf{基于单目视觉的同时定位与地图构建方法}
\end{center}
\author{吴\ 王\ 朱}
\section{Induction}
\subsection{增强现实技术}
增强现实技术(\textbf{augmented reality})是一种将真实世界信息和虚拟世界信息“无缝”集成的新技术,是把原本在现实世界的一定时间空间范围内很难体验到的实体信息(视觉信息,声音,味道,触觉等),通过电脑等科学技术,模拟仿真后再叠加,将虚拟的信息应用到真实世界,被人类感官所感知,从而达到超越现实的感官体验。比起传统方式来说,它更加的直观,更加的高效,因此也有着更加广阔的应用前景。近年来,增强现实技术已在军事,生活,游戏等众多领域运营并取得了成功。例如,宜家家居公司已经开发了一个APP使得用户可以使用智能手机观察不同的家具在自己房间的摆放效果;而任天堂公司也开发了Pokemon-Go 游戏,使得玩家可以通过智能手机在现实世界里发现精灵。

\subsection{增强现实的定位方式}
增强现实需要实时定位设备在环境中的方位,定位的方案虽然有许多种,但多数方案都存在局限或者代价太高难以普及,例如GPS无法在室内及遮挡严重的环境里使用,且精度较低,而基于无线信号的定位方案则需要事先布置场景。基于视觉的同时定位与地图构建技术(\textit{visual simultaneous localization and mapping} \textbf{V-SLAM})以其成本低廉、小场景精度较高、无需预先布置场景等优势成为比较常采用的定位方案。

\subsection{V-SLAM技术}
V-SLAM 技术指的是使用图像作为外部信息的唯一来源,来定位一个机器人、一辆车或者一个移动的相机在整个场景中的位置,同时,重建环境的三维结构。


\subsubsection{V-SLAM的基本原理}
V-SLAM技术根据拍摄的视频、图像信息推断摄像头在环境的方位,同时构建环境地图,其原理为多视图几何原理(\textbf{Multiple view geometry theroy}) V-SLAM的目标为同时恢复出每帧图像对应的相机运动参数$C_1$,$C_2$ … $C_m$以及场景三维结构$X_1$,$X_2$ … $X_n$,每个相机运动参数$C_i$包含了相机的位置和朝向信息,通常表达为一个 3$\times$3的旋转矩阵$R_i$和一个三维位置变量 $p_i$。$R_i$与$p_i$将一个世界坐标系下的三维点$X_j$变换至$C_i$的局部坐标系
\begin{equation}\label{EQ1}
{(X_{ij},Y_{ij},Z_{ij})}^\mathrm{T}=R_{i}(X_{j}-p_{i})
\end{equation}
进而投影至图像中
\begin{equation}\label{EQ2}
h_{ij}={(f_{x}X_{ij}/Z_{ij}+c_{x}.f_{y}Y_{ij}/Z_{ij}+c_{y})}^\mathrm{T}
\end{equation}
其中,$f_x$,$f_y$分别为沿图像$x$,$y$轴的图像焦距,$(c_x,c_y)$为镜头光心在图像中的位置,通常假设这些参数已实现标定且保持不变,由式
\eqref{EQ1} \eqref{EQ2},三维点在图像中的投影位置$h_{ij}$可表示为一个关于$C_i$和$X_j$的函数,记为
\begin{equation}\label{EQ3}
h_{ij}=h(C_i,X_j)
\end{equation}
V-SLAM算法需要将、对不同图像中对应于相同场景的图像点进行匹配,而这个过程是通过求解如下目标函数
\begin{equation}\label{EQ4}
\mathop{\arg\min}_{C_1,…C_m,X_1,…X_n} \sum_{i=1}^{m} {\sum_{j=1}^{n} \ \| {h} (C_i,X_j)-\tilde{x_{ij}}\|_{\Sigma_{ij}}}
\end{equation}
得到一组最优的$C_1$,$C_2$…$C_m$,$X_1$,$X_2$ … $X_n$,使得所有$X_j$在$C_i$图像中的投影位置$h_{ij}$与观测到的图像点位置$x_{ij}$尽可能靠近,这里假设图像观测点符合高斯分布$x_{ij}$ $\sim$ $\mathrm{\textit{N}}$ ($\tilde{x_{ij}}$,$\Sigma_{ij}$),$\|$ $e$ $\|$ =$e^{\mathrm{T}}$ $\Sigma^{-1}$ $e$
求解目标函数\eqref{EQ4}的过程也成为集束调整(\textbf{bundle adjustment,BA}),该最优化问题可利用线性方程的稀疏结构高效求解。
\subsubsection{主流的单目V-SLAM方法}
由于现阶段大多数AR产品都以智能手机以及平板电脑作为载体,而智能手机的摄像头大多以单目为主,双目、三目摄像头甚至深度摄像头都未得到普及,因此本文主要讨论基于单目视觉的同时定位与地图构建方法。
目前,主流的V-SLAM方法主要为:基于滤波器、基于关键帧BA和基于直接跟踪,我们先来看看这三种方法。比较并分析其优劣,而后详细介绍基于关键帧BA的V-SLAM方法。其中比较具有代表性的有MonoSLAM以及MSCKF。
基于滤波器的V-SLAM的方法将系统每一时刻的状态\textit{t}用一个高斯概率模型表达,$x_t$$\sim$ $\mathrm{\textit{N}}$ ($\tilde{x_{t}}$,$\mathrm{\textit{P}}_{ij}$),其中$\tilde{x}_t$为当前时刻系统状态估计值,$P_t$为该估计值误差的协方差矩阵,系统状态由滤波器不断更新。
而基于关键帧BA的V-SLAM方法是近年来最流行的方法之一,他的主要思想是将相机跟踪(Tracking)和地图构建(Mapping)作为两个独立的任务在两个线程并行执行,而Mapping线程仅维护视频流中抽取的关键帧。PTAM是最著名的基于关键帧BA的方法之一,也是我们介绍的重点。
基于直接跟踪的V-SLAM方法则是直接通过比较像素颜色来求解相机运动,具有代表性的算法有DTAM以及LSD-SLAM。
\subsubsection{不同V-SLAM方法的比较}
比较三种方法,基于直接跟踪的V-SLAM方法对光照变化和动态干扰较为敏感,所以精度会若于基于关键帧BA的V-SLAM方法,而基于滤波器的V-SLAM由于容易在变量未精确的时候就进行消元,误差不断累积,因此精度最差。而在定位效率与场景尺度方面,基于滤波器的V-SLAM方法计算复杂度较高,只适用于几百个点的小场景,因此定位效率和场景尺度较低,而基于关键帧的V-SLAM方法例如PTAM,它只优化当前帧方位就因此具有最高的定位效率,但场景尺度受到bundle adjustment 的限制因此也不高,而基于直接跟踪的V-SLAM方法定位效率取决于图像,分辨率,且采用更高效的方位图优化替代全局BA,因此具有最高的场景尺度限制。
总的来说,基于关键帧的方法具有更好的V-SLAM效果。


\end{document}